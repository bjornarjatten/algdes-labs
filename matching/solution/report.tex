\documentclass{tufte-handout}
\usepackage{amsmath}
\usepackage[utf8]{inputenc}
\usepackage{mathpazo}
\usepackage{booktabs}
\usepackage{microtype}

\pagestyle{empty}


\title{Stable Matching Report}
\author{Bjarke Brodin (bjal), Bjørnar Haugstad Jåtten (bjja), Helle Friis (href) & Simon Boye Jørgensen (sboj)}

\begin{document}
  \maketitle

  \section{Results}

  Our implementation produces the expected results on all input--output file pairs, except {\tt sm-random-100.txt}, where it matches 54 with 12 instead of 2.
  We have no idea why this happens.%
  \sidenote{%
  Complete the report by filling in your correct names,
  filling in the parts marked $[\ldots]$,
  and changing other parts wherever necessary.
  For instance, if your implementation passes all tests, then write that.
  Remove the sidenotes in your final hand-in.
  }

  On input {\tt sm-bbt-in.txt}, we produce the following matching:
  \begin{quotation}
    Sheldon--Amy, Rajesh--Penny, Howard--Bernadette, Leonard--Priya.  \sidenote{Replace with your results.}
  \end{quotation}

  \section{Implementation details}

  The men's preferences are stored in a HashMap, where the key is the ID of the specific man and the value is an ArrayDeque composed of their preferences.
  The women's preferences are stored in a HashMap where the key is the ID of the specific woman. 
  The value is another HashMap where the key is the ID of the man and the value is the rank of given by the woman to the man.

  We can find a free man who has not proposed to every woman in time $O(1)$,
  because we store free men in an ArrayDeque.

  With these data structures, our implementation runs in time $O(n^2)$ on inputs with $n$ men and $n$ women.


\end{document}
