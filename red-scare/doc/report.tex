\documentclass{tufte-handout}
\usepackage[utf8]{inputenc}
\usepackage{tikz}
\usepackage{amsmath}

\usepackage{color}
\newcommand{\red}[1]{{\color{red} #1}}
\renewcommand{\arraystretch}{0.4}
\usepackage{booktabs}

\title{Red Scare! Report}
\author{
  Bjarke Brodin (bjal)\\
  Bjørnar Haugstad Jåtten (bjja)\\
  Helle Friis (hefr)\\
  and Simon Boye Jørgensen (sboj)
}

\begin{document}

\maketitle

\section{Results}

The following table gives my results for all graphs of at least 500 vertices.

\medskip
\begin{tabular}{lrrrrrr}
  \toprule
  Instance name & $n$ & A & F & M & N & S \\
  \midrule
  bht.txt&	5757 &	false &	-1 &	?! &	6 &	?! &	\\
  common-1-1000.txt&	1000 &	false &	-1 &	-1 &	-1 &	false &	\\
  common-1-1500.txt&	1500 &	false &	-1 &	-1 &	-1 &	false &	\\
  common-1-2000.txt&	2000 &	false &	-1 &	-1 &	-1 &	false &	\\
  common-1-2500.txt&	2500 &	false &	1 &	?! &	6 &	?! &	\\
  common-1-3000.txt&	3000 &	false &	1 &	?! &	6 &	?! &	\\
  common-1-3500.txt&	3500 &	false &	1 &	?! &	6 &	?! &	\\
  common-1-4000.txt&	4000 &	false &	1 &	?! &	6 &	?! &	\\
  common-1-4500.txt&	4500 &	true &	1 &	?! &	6 &	?! &	\\
  common-1-500.txt&	500 &	false &	-1 &	-1 &	-1 &	false &	\\
  common-1-5000.txt&	5000 &	true &	1 &	?! &	6 &	?! &	\\
  common-1-5757.txt&	5757 &	true &	1 &	?! &	6 &	?! &	\\
  common-2-1000.txt&	1000 &	true &	1 &	?! &	4 &	?! &	\\
  common-2-1500.txt&	1500 &	true &	1 &	?! &	4 &	?! &	\\
  common-2-2000.txt&	2000 &	true &	1 &	?! &	4 &	?! &	\\
  common-2-2500.txt&	2500 &	true &	1 &	?! &	4 &	?! &	\\
  common-2-3000.txt&	3000 &	true &	1 &	?! &	4 &	?! &	\\
  common-2-3500.txt&	3500 &	true &	1 &	?! &	4 &	?! &	\\
  common-2-4000.txt&	4000 &	true &	1 &	?! &	4 &	?! &	\\
  common-2-4500.txt&	4500 &	true &	1 &	?! &	4 &	?! &	\\
  common-2-500.txt&	500 &	true &	1 &	?! &	4 &	?! &	\\
  common-2-5000.txt&	5000 &	true &	1 &	?! &	4 &	?! &	\\
  common-2-5757.txt&	5757 &	true &	1 &	?! &	4 &	?! &	\\
    \vdots
  \bottomrule
\end{tabular}

\begin{tabular}{lrrrrrr}
  \toprule
  Instance name & $n$ & A & F & M & N & S \\
  \midrule
  gnm-1000-1500-0.txt&	1000 &	false &	1 &	?! &	-1 &	?! &	\\
  gnm-1000-1500-1.txt&	1000 &	false &	2 &	?! &	-1 &	?! &	\\
  gnm-1000-2000-0.txt&	1000 &	false &	-1 &	?! &	7 &	?! &	\\
  gnm-1000-2000-1.txt&	1000 &	false &	1 &	?! &	-1 &	?! &	\\
  gnm-2000-3000-0.txt&	2000 &	false &	-1 &	?! &	8 &	?! &	\\
  gnm-2000-3000-1.txt&	2000 &	true &	2 &	?! &	-1 &	?! &	\\
  gnm-2000-4000-0.txt&	2000 &	false &	-1 &	?! &	6 &	?! &	\\
  gnm-2000-4000-1.txt&	2000 &	false &	-1 &	?! &	5 &	?! &	\\
  gnm-3000-4500-0.txt&	3000 &	false &	-1 &	?! &	10 &	?! &	\\
  gnm-3000-4500-1.txt&	3000 &	false &	2 &	?! &	-1 &	?! &	\\
  gnm-3000-6000-0.txt&	3000 &	false &	-1 &	?! &	6 &	?! &	\\
  gnm-3000-6000-1.txt&	3000 &	false &	1 &	?! &	6 &	?! &	\\
  gnm-4000-6000-0.txt&	4000 &	false &	-1 &	?! &	7 &	?! &	\\
  gnm-4000-6000-1.txt&	4000 &	false &	-1 &	?! &	15 &	?! &	\\
  gnm-4000-8000-0.txt&	4000 &	false &	-1 &	?! &	5 &	?! &	\\
  gnm-4000-8000-1.txt&	4000 &	true &	1 &	?! &	6 &	?! &	\\
  gnm-5000-10000-0.txt&	5000 &	false &	1 &	?! &	5 &	?! &	\\
  gnm-5000-10000-1.txt&	5000 &	true &	-1 &	?! &	5 &	?! &	\\
  gnm-5000-7500-0.txt&	5000 &	false &	-1 &	-1 &	-1 &	false &	\\
  gnm-5000-7500-1.txt&	5000 &	false &	-1 &	-1 &	-1 &	false &	\\
  grid-25-0.txt&	625 &	true &	-1 &	?! &	324 &	?! &	\\
  grid-25-1.txt&	625 &	true &	-1 &	?! &	123 &	?! &	\\
  grid-25-2.txt&	625 &	true &	5 &	?! &	-1 &	?! &	\\
  grid-50-0.txt&	2500 &	false &	-1 &	?! &	1249 &	?! &	\\
  grid-50-1.txt&	2500 &	false &	-1 &	?! &	521 &	?! &	\\
  grid-50-2.txt&	2500 &	false &	10 &	?! &	-1 &	?! &	\\
  increase-n500-1.txt&	500 &	true &	1 &	15 &	1 &	true &	\\
  increase-n500-2.txt&	500 &	true &	1 &	17 &	1 &	true &	\\
  increase-n500-3.txt&	500 &	true &	1 &	16 &	1 &	true &	\\
  rusty-1-2000.txt&	2000 &	false &	-1 &	-1 &	-1 &	false &	\\
  rusty-1-2500.txt&	2500 &	false &	-1 &	-1 &	-1 &	false &	\\
  rusty-1-3000.txt&	3000 &	false &	-1 &	?! &	14 &	?! &	\\
  rusty-1-3500.txt&	3500 &	false &	-1 &	?! &	14 &	?! &	\\
  rusty-1-4000.txt&	4000 &	false &	-1 &	?! &	13 &	?! &	\\
  rusty-1-4500.txt&	4500 &	false &	-1 &	?! &	7 &	?! &	\\
  rusty-1-5000.txt&	5000 &	false &	-1 &	?! &	7 &	?! &	\\
  rusty-1-5757.txt&	5757 &	false &	-1 &	?! &	7 &	?! &	\\
    \vdots
  \bottomrule
\end{tabular}

\begin{tabular}{lrrrrrr}
  \toprule
  Instance name & $n$ & A & F & M & N & S \\
  \midrule
  rusty-2-2000.txt&	2000 &	false &	-1 &	?! &	5 &	?! &	\\
  rusty-2-2500.txt&	2500 &	false &	-1 &	?! &	4 &	?! &	\\
  rusty-2-3000.txt&	3000 &	false &	-1 &	?! &	4 &	?! &	\\
  rusty-2-3500.txt&	3500 &	false &	-1 &	?! &	4 &	?! &	\\
  rusty-2-4000.txt&	4000 &	false &	-1 &	?! &	4 &	?! &	\\
  rusty-2-4500.txt&	4500 &	false &	-1 &	?! &	4 &	?! &	\\
  rusty-2-5000.txt&	5000 &	false &	-1 &	?! &	4 &	?! &	\\
  rusty-2-5757.txt&	5757 &	false &	-1 &	?! &	4 &	?! &	\\
  smallworld-30-0.txt&	900 &	false &	-1 &	?! &	9 &	?! &	\\
  smallworld-30-1.txt&	900 &	true &	-1 &	?! &	11 &	?! &	\\
  smallworld-40-0.txt&	1600 &	false &	-1 &	?! &	8 &	?! &	\\
  smallworld-40-1.txt&	1600 &	true &	-1 &	?! &	13 &	?! &	\\
  smallworld-50-0.txt&	2500 &	false &	-1 &	?! &	3 &	?! &	\\
  smallworld-50-1.txt&	2500 &	true &	1 &	?! &	-1 &	?! &	\\
  wall-n-100.txt&	800 &	false &	-1 &	?! &	1 &	?! &	\\
  wall-n-1000.txt&	8000 &	false &	-1 &	?! &	1 &	?! &	\\
  wall-p-100.txt&	602 &	false &	-1 &	?! &	1 &	?! &	\\
  wall-p-1000.txt&	6002 &	false &	-1 &	?! &	1 &	?! &	\\
  wall-p-10000.txt&	60002 &	false &	-1 &	?! &	1 &	?! &	\\
  wall-z-100.txt&	701 &	false &	-1 &	?! &	1 &	?! &	\\
  wall-z-1000.txt&	7001 &	false &	-1 &	?! &	1 &	?! &	\\
  wall-z-10000.txt&	70001 &	false &	-1 &	?! &	1 &	?! &	\\
  \bottomrule
\end{tabular}
\medskip

%   \vdots
%   \bottomrule
% \end{tabular}

% \begin{tabular}{lrrrrrr}
%   \toprule
%   Instance name & $n$ & A & F & M & N & S \\
%   \midrule

The columns are for the problems Alternate, Few, Many, None, and Some.
The table entries either give the answer, or contain `?' for those cases where we was unable to find a solution within reasonable time.
For those questions where there is a reason for our inability to find a good algorithm (because the problem is hard), we wrote `?!'.

For the complete table of all results, see the tab-separated text file {\tt results.txt}.

\section{Methods}

For all solutions we process the input to produce a graph $G$,
which we assume to be simple.

We preprocess each instance filtering out any irellevant components,
and even return immediately if $s$ and $t$ are in different components.
This is rather trivial to do in $O(|V|+|E| \log |E|)$ e.g. by using union-find.

\subsection{None}
We solve this problem efficiently for all graphs.
Here we did not put any of the red vertex in the graph.
Then when running the shortest path algorithm we were guaranteed
to find the shortest path without any red vertexes in the path
if there exists one.

We use BFS for the shortest path algorithm which has a 
running time of $O((|E| + |V|) \log |V|)$.

\subsection{Alternate}
To solve alternate, we first removed all reflective edges, 
meaning the edges that go from black to black or red to red.
Afterwards, we applied a shortest path algorithm.
If there exists a path between t and s, it must be 
alternating between black and red, as all the relfective edges are removed from the graph.

Running time is dominated by shortest path and is thus simply
$O((|E|+|V|) \log |V|)$.

\subsection{Few}
To solve \textit{few}, 
we realize that we can manipulate edge weights
to allow a reduction to shortest path.
We set weights to 1 for all arcs that 
are incident on a red vertex and all other weights to 0,
then simply run an appropriate shortest path algorithm 
and return the distance.
Running time is dominated by shortest path and is thus simply
$O((|E|+|V|) \log |V|)$.

\subsection{Many}
We solve many for directed acyclic graphs,
by using much the same approach as in few,
only, instead of setting 1 weighted red arcs,
we use -1 weighted red arcs.
Running shortest path then yields the negation
of our desired result.
We take care to use a shortest path algorithm
that does not stop early when reaching a valid path,
but rather examines all options.

In the general case we cannot solve many,
because the general case includes $R = G$.
Solving many in this case implies solving 
the longest path problem exactly, 
and since this is known to be NP-hard
we can safely give up. %TODO refer to KT

\subsection{Some}
First we note that solving many implies solving some,
since many $\ge$ 1 is equivalent to the result of some. \\

However we can utilize the maximum flow algorithm to solve
some in polynomial time even when we may not be able to solve many.
We construct an $O(|V||E||R|)$ solution as follows:
\begin{itemize}
  \item Ensure that all edges in the graph are bidirectional by filtering any that aren't, set all weights to 1.
  \item Ensure that each vertex $v$ can only be used once, by splitting it into vertices $v$ and $|V|+v$ such that all incoming edges go to
  $v$ and all outgoing edges go from $|V|+v$. Then add a directional edge from $v$ to $|V|+v$ with weight 1.
  \item Add a new source vertex $s'$ and two new edges from $s'$ to $s$ and from $s'$ to $t$ both with weight 1.
  \item For each red vertex $r$ compute maxflow$(s', r)$ -- if any flow is 2
    then some must be 'true' (because every used original edge is
    bidirectional), in fact any maxflow is at most 2 which reduces the worst
    case running time considerably.
    If no such flow is found we can't conclude that 'false' is the 
    correct result, since we don't have sufficient information to make this claim.
\end{itemize}

In the general case the best solution we know of is 
enumerating all s-t paths, which is an exponential time algorithm,
thus we give up in the general case.


\section{References}
\begin{description}

\end{description}

\end{document}
